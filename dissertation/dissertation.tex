% dissertation.tex	K. J. Turner	10/09/13

% Dissertation outline

\documentclass[a4paper,11pt]{report}

\usepackage{dissertation}
\usepackage{identfirst}
\usepackage{graphicx}
\usepackage{chngcntr}
\counterwithout{figure}{chapter}

\title{Smart contracts on Hyperledger Fabric}			

\author{
{Yordan Gospodinov} \\
{\small{Supervised by Dr. Andrea Bracciali}}
}

\renewcommand{\baselinestretch}{1.1}
\date{April 2019}			

\degree{B. Sc. in Computer Science}

\begin{document}

\maketitle				

\pagenumbering{roman}


%%%%%%%%%%%%%%%%%%%%%%%%%%%%%%%%%%  Abstract %%%%%%%%%%%%%%%%%%%%%%%%%%%%%%%%%%%

\intro{Abstract}

Summarise the dissertation within one page. Introductory headings like this are
entered using the {\it intro} paragraph style. It is suggested that the abstract
be structured as follows:

\begin{description}

  \item[Problem:]
  what you tackled, and why this needed a solution

  \item[Objectives:]
  what you set out to achieve, and how this addressed the problem

  \item[Methodology:]
  how you went about solving the problem

  \item[Achievements:]
  what you managed to achieve, and how far it meets your objectives.

\end{description}


%%%%%%%%%%%%%%%%%%%%%%%%%%%%%%%%%  Attestation %%%%%%%%%%%%%%%%%%%%%%%%%%%%%%%%%

\intro{Attestation}

I understand the nature of plagiarism, and am aware of the University's policy
on this. I certify that this dissertation reports original work by me during my
University project except for the following (adjust according to the
circumstances):

\begin{itemize}

  \item
  The technology review in section~\ref{technical-section2} was largely adapted
  %from \cite{Greene-Williams-1997} plus
  \url{www.software-review.org/article9815.html}.

  \item
  The code discussed in section~\ref{technical-section1} was created by Acme
  Corporation (\url{www.acme-corp.com/JavaExpert}) and was used in accordance
  with the licence supplied.

  \item
  The code discussed in section~\ref{technical-subsection1} was written by my
  supervisor.

  \item
  The code discussed in section~\ref{technical-subsubsection2} was developed by
  me during a vacation placement with the collaborating company. In addition,
  this used ideas I had already developed in my own time.

\end{itemize}

\bigskip

{\bf Signature:} \hspace{20em} {\bf Date:}


%%%%%%%%%%%%%%%%%%%%%%%%%%%%%%%  Acknowledgements %%%%%%%%%%%%%%%%%%%%%%%%%%%%%%

\intro{Acknowledgements}

First and foremost I would like to thank to Dr. Andrea Bracciali for introducing me to blockchain. What's more I am grateful for his patience, guidance and support, without which I would not have been able to grow into what I am today. Thank you.

I am grateful to my parents for giving me the opportunity to study in University of Stirling. 

I am grateful to my classmates for helping me out, when I could not understand something, and just for being around, it was fun. 

I would like to thank the library for having the nice, quiet, fourth floor, and for always buying the books I requested. 

%%%%%%%%%%%%%%%%%%%%%%%%%%%%%%%%%%  Contents %%%%%%%%%%%%%%%%%%%%%%%%%%%%%%%%%%%

\tableofcontents

\listoffigures                

% \listoftables                 

\clearpage

\pagenumbering{arabic}

\setcounter{page}{1}


%%%%%%%%%%%%%%%%%%%%%%%%%%%%%%%%  Introduction %%%%%%%%%%%%%%%%%%%%%%%%%%%%%%%%%

\chapter{Introduction}

\label{introduction}

Recently a new technology emerged into the world that is called blockchain. The idea is that it is a distributed decentralized database of blocks of transactions. The reason why I find it so interesting is that it has the potential to change many aspects of life, by changing the way of accessibility and security of a data. Hyperledger Fabric is a blockchain framework designed especially to make the system easily adoptable for different business use cases. The main two objectives of this project are: to research how the smart contracts in Hyperledger Fabric works; to make a prototype of self-sovereign system with the gathered knowledge. 

\section{Background and Context}

\label{introduction-background}

Information has always been one of the most valuable assets a person could have. Through times information was traded in many different ways, from barter to monetization. Recently the information about an individual has become a great selling point, because it can be used in variety of fields, from science to business. However, the collection of this data is becoming a problem.

	As individuals, our identities are, to some extend, not ours anymore. If we cannot certify who we are, we became no one in the eyes of business and government. Needless to say that have we lost all of the documents that certify our place in the city, company, country, Earth, we would be in a big trouble. \cite{allen2016path}
	
		Another approach to critical and private information is how it is being used live. Whenever we want to identify ourselves somewhere, the usual document for identification would be either an ID or a passport. Here is the problem concerning all information on this document. It turns out that whenever a person wants to prove his or her existence, the party that requires this identification, can take and keep a record of all sensitive data on that document. In some countries this may be illegal. This data then could be used for not a rightful purpose. \cite{alboaie2017private}
		
Furthermore whenever a person is signing in to receive some kind of certificate, whether that would be a school or an academy, he or she is leaving sensitive data with this company. In most countries, whenever a person starts living in a city, he or she has to identify himself/herself to the council. In the end, there is a lot of institutions that keep sensitive data for an individual. This is a problem, because some of those institutions or businesses have different levels of security. So, an attacker only needs to pick the easiest target, and he will get a great deal of sensitive data. 

	I believe all of these problems are just a subproblems of a bigger challenges - what is an identity today and how to be able to give private access to our data. The solution could provide us awareness for a better control of our own data, as well as to be able to share only what’s exactly needed to provide to those companies and institutions.


\section{Scope and Objectives}

\label{introduction-objectives}

The scope will involve Blockchain technology and what is digital identity. This project will focus on Hyperledger Fabric. This is a permissionless blockchain modular framework, especially developed for businesses.
 
Being a modular framework, a lot of the scope will involve around resolving how customizable Fabric can be. To be personalized is of an essence for the creation of a good system. The other main features to be examined are the scalability and usability of this blockchain framework.   

The knowledge build up from the research will be implemented in a prototype program as a final part of the project. The prototype of the self-sovereign program will focus on the decentralized nature. This work will aim to present advantages of the decentralizing element that can save resources and protect the personal data of the end-user. Last but not least, I am going to talk about how the ledger is making the whole system trustful, thus no one of the parties needs to worry about being cheated. 
 	
The objectives of the project include the following :  
\begin{itemize}
\item Understanding how Hyperledger Fabric work;
	\begin{itemize}
	\item Installing all prerequisites;
	\item Installing Hyperledger Fabric; 
	\item Running a simple network with 2 organizations; 
	\item Learning how to add more parties into an already running system;
	\item Trying out how the chaincode (smart contracts) work;
	\item Trying to install and control newly added chaincode on a running system.
	\end{itemize}
\item Building a fully functional Fabric blockchain with several different parties;
\item What an ID is  and identity and how it is defined in the digital world;
\item Deeper understanding of self-sovereign identity, what it is and how it should/could be best defined in a blockchain platform in order to be used genuinely and without misappropriation;
	\begin{itemize}
	\item Trying out different configurations on Fabric;
	\item Trying out different chaincode functions, to find out the best for the use case.
	\end{itemize}
\item Building a prototype of self-sovereign identity system ;
\item Complete final report .

\end{itemize}


\section{Achievements}

\label{introduction-achievements}

Summarise what you have achieved.

\section{Overview of Dissertation}

Briefly overview the contents of what follows in the dissertation.


%%%%%%%%%%%%%%%%%%%%%%%%%%%%%%  State-of-The-Art %%%%%%%%%%%%%%%%%%%%%%%%%%%%%%%

\chapter{State-of-The-Art}

\label{state}
\section{Successful projects made with Hyperledger Fabric}
\label{successfulFabric}

\subsection{Altoros}
\label{altoros}
Altoros is a software company that delivers different solutions. One of the problems their customers have is issuing bonds. The customer, Russia's National Settlement Depository (NSD), wanted a system that allows automate bond placement and accounting with blockchain, while minimizing risks of reconciliation and ensuring transparency. The reason they chose Fabric is for its support of confidential transactions and resilience in the production environment. \cite{altoros}

	What they did was to customize Fabric as needed for the different roles and actions. They set up four different channels so the communication,data transferring, between the peers and the NSD could be safe and secure. Every channel has its own chaincode ( smart contract) that is basically the logistics behind the given channel.
	
	One of the challenges they had was that the REST API was still in development. Fortunately, this is not the case anymore. Another challenge is that Fabric does not support cross-channel transactions. \cite{altorosDemo}
	
	The benefits of choosing Fabric are: 
	\begin{itemize}
	
\item Faster transactions compared to the traditional solution, where a lot of data exchanging has to be done through a middleman. Thus, not only making it faster but also cheaper. 
\item Minimizing fraud in a secure trusted network. The permissioned feature does not allow for anyone that does not meet the requirements to monitor what’s happening into the world ledger. What’s more because of the non cross-channel transactions, a peer could observe only the channels he is using. And even when he or she is inspecting another peer’s transaction, because of the encryption, he or she would not get any valuable information. 
\item Reduces expenses of the bond issuer by making the process faster and simplified
	\end{itemize}

\subsection{Verify.Me}
\label{verifyMe}
SecureKey is a company providing identity and authentication provider for simplified access to online services and applications. They are using trusted providers such as banks, telcos and governments to make their clients assert identity information and connect to critical online services with digital credentials.

After the government of Canada recognized their problem sending private data to a citizen, they asked for a solution. SecureKey responded to this call in collaboration with IBM with a blockchain based solution. It is a mobile app, that allows the user to connect different types of services providing only specific data. So what happens is the user connects to the blockchain through the phone. Then, it connects with the service actors. It is important to note that in the phone there are only pointers to the data and not the data itself. Whenever a person is sharing his or her identity with the new service he or she can see exactly what information is asked to be provided. \cite{verifyMe}

	The SIM card is used as an anchor of trust. Since the system is
private and permissioned blockchain, only trusted actors like banks and government can write on it. Upon losing or breaking the phone, the creators reassure that is easy to recover what’s lost. 
Again, here one of the main reasons to choose Fabric for the development of this service is mainly - the adaptability of the platform and the zero-knowledge proof supported concept. \cite{verifyMeDemo}

The benefits of using Fabric are : 
\begin{itemize}
\item Data integrity 
\item Security and resiliency
\item No central database or honeypots 
\item No central point of failure
\item Cannot track user across relying parties; privacy of the data
\item Cost efficient due to simplifying the process
\end{itemize}
Cons: 
\begin{itemize}
\item New - open standards needed 
\end{itemize}

\subsection{TradeLens}
\label{tradelens}

	TradeLens is a company founded by collaborative work of Maersk and IBM. Maersk is an integrated container logistics company working on improving the supply chain area. The idea is to make the shipping process cost-efficient, faster and in respect to accessing the needed documents - simpler.
	 
	For this task, the collaboration is combining their technical and specialized knowledge to build a system on top of Hyperledger Fabric. What they created is a network, that tracks the supply chain - the documents needed for starting a shipping process, the deal that is made, the location of the containers.
	 
	To participate, a user has to pay a price to enter the network. Still it is not confirmed what the requirements are. However, once a user decides to enter he will experience something way different from the usual way of things. Due to the blockchain technology, a user can check a block on the blockchain to track the location of the container or any other process involved. The usual way for this simple task would be to request this information from a middleman. TradeLens are saying they can reduce the paperwork and the need of a mediators, saving lots of time and money in the process. \cite{tradeLensFounders}
	
	It is important to be mentioned that TradeLens is not fighting the frauds. If a user input false data at start,that seems to be correct to the endorsement parties, the system won’t be able to catch it. So the network helps to have less fraud, but it is more of a side effect rather than main function.
	
	Another great use of this system is that, according to the World Trade Organization, simplifying the supply chain will not only reduce costs, but also help developing countries to increase their export by more than 30\% . [16]
	
\subsection{BitNation}
\label{bitnation}


\subsection{E-Residency}
\label{eResidency}

%%%%%%%%%%%%%%%%%%%%%%%%%%%%%% Technical Chapters %%%%%%%%%%%%%%%%%%%%%%%%%%%%%%

\chapter{Technical Chapters}            % Insert your chapter titles

\label{technical}

The body of the dissertation consists of a number of chapters named
appropriately ({\em not} `Technical Chapter'). Follow a logical progression in
how you present your work. This might be a time sequence of development
activities, the phases of the software development cycle, the modules of your
system, etc.

Appropriate chapters might be called Requirements, Design, Implementation and
Evaluation. The emphasis should be on requirements, design and evaluation, with
implementation details being of lesser importance. The requirements should be
clearly stated, following from the client needs and weaknesses identified in the
state-of-the-art review. The design should include discussion of the choices
that were available and why particular decisions were made. The evaluation
should relate back to the requirements, and demonstrate the extent to which
these were met. Low-level material should appear in appendixes.

\section{First Section}

\label{technical-section1}

Subdivide your text into sections with the \textit{{\bs}section} command.

\subsection{First Subsection}

\label{technical-subsection1}

If necessary, also use subsections. Subsections are entered using the
\textit{{\bs}subsection} command.

\subsubsection{First Subsubsection}

\label{technical-subsubsection1}

If you really need subsubsections, enter these using the
\textit{{\bs}subsubsection} command.

\subsubsection{Second Subsubsection}

\label{technical-subsubsection2}

And yet more subsubsections if need be.

\subsection{Second Subsection}

\label{technical-subsection2}

And, as required, more subsections.

\section{Second Section}

\label{technical-section2}

Figures are created with the \textit{figure} environment, while tables are
created with the \textit{table} environment. They are identified by the
\textit{{\bs}label} command, and are referenced by the \textit{{\bs}ref}
command. Graphics are inserted with the \textit{{\bs}graphic} command. Captions
are entered using the \textit{{\bs}caption} command. As an example of a figure,
consider figure~\ref{figure}.

The native format for \LaTeX\ graphics is EPS (Encapsulated PostScript).
Graphical editors are usually capable of producing EPS. When outputting to PDF
(Portable Document Format), the native graphics format is also PDF. Conversion
of EPS to PDF is supported by a number of TeX\ toolsets.

\begin{figure}

  \graphic{figure}

  \caption{Highly Technical Diagram}

  \label{figure}

\end{figure}


%%%%%%%%%%%%%%%%%%%%%%%%%%%%%%%%%%  Conclusion %%%%%%%%%%%%%%%%%%%%%%%%%%%%%%%%%

\chapter{Conclusion}

\label{conclusion}

\section{Evaluation}

\label{conclusion-evaluation}

If you do not have a separate chapter on testing, explain here in detail how you
went about systematically testing your system. If appropriate, also include
end users in your testing. Summarise your main results, and explain how you have
advanced the state-of-the-art. Stand back and evaluate what you have achieved
and how well you have met the objectives. Evaluate your achievements against the
objectives stated in section~\ref{introduction-objectives}. Demonstrate that you
have tackled the project in a professional manner.

\section{Future Work}

\label{conclusion-future}

Explain any limitations in your results and how things might be improved.
Discuss how your work might be developed further. Reflect on your results in
isolation and in relation to what others have achieved in the same field. This
self-analysis is particularly important. You should give a critical evaluation
of what went well, and what might be improved.

% Citations

\bibliographystyle{abbrv}

\bibliography{ref}

% Appendixes

\appendix


%%%%%%%%%%%%%%%%%%%%%%%%%%%%%%%%%%  User Guide %%%%%%%%%%%%%%%%%%%%%%%%%%%%%%%%%

\chapter{User Guide}			% Insert your appendix titles

\label{guide}

The appendixes should contain reference material or detailed material that would
detract from the flow in the body of the dissertation. Appendixes might include
a user guide, a list of abbreviations, detailed program descriptions, etc.
Appendixes are introduced with the \textit{{\bs}appendix} command. Appendix
headings otherwise use \textit{{\bs}chapter}, \textit{{\bs}section}, etc.\ as
usual.

\end{document}
