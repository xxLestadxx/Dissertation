% dissertation.tex	K. J. Turner	10/09/13

% Dissertation outline

\documentclass[a4paper,11pt]{report}

\usepackage{dissertation}

\title{Smart contracts on Hyperledger Fabric}			

\author{Yordan Gospodinov}			

\date{April 2019}			

\degree{B. Sc. in Computer Science}

\begin{document}

\maketitle				

\pagenumbering{roman}


%%%%%%%%%%%%%%%%%%%%%%%%%%%%%%%%%%  Abstract %%%%%%%%%%%%%%%%%%%%%%%%%%%%%%%%%%%

\intro{Abstract}

Summarise the dissertation within one page. Introductory headings like this are
entered using the {\it intro} paragraph style. It is suggested that the abstract
be structured as follows:

\begin{description}

  \item[Problem:]
  what you tackled, and why this needed a solution

  \item[Objectives:]
  what you set out to achieve, and how this addressed the problem

  \item[Methodology:]
  how you went about solving the problem

  \item[Achievements:]
  what you managed to achieve, and how far it meets your objectives.

\end{description}


%%%%%%%%%%%%%%%%%%%%%%%%%%%%%%%%%  Attestation %%%%%%%%%%%%%%%%%%%%%%%%%%%%%%%%%

\intro{Attestation}

I understand the nature of plagiarism, and am aware of the University's policy
on this. I certify that this dissertation reports original work by me during my
University project except for the following (adjust according to the
circumstances):

\begin{itemize}

  \item
  The technology review in section~\ref{technical-section2} was largely adapted
  from \cite{Greene-Williams-1997} plus
  \url{www.software-review.org/article9815.html}.

  \item
  The code discussed in section~\ref{technical-section1} was created by Acme
  Corporation (\url{www.acme-corp.com/JavaExpert}) and was used in accordance
  with the licence supplied.

  \item
  The code discussed in section~\ref{technical-subsection1} was written by my
  supervisor.

  \item
  The code discussed in section~\ref{technical-subsubsection2} was developed by
  me during a vacation placement with the collaborating company. In addition,
  this used ideas I had already developed in my own time.

\end{itemize}

\bigskip

{\bf Signature:} \hspace{20em} {\bf Date:}


%%%%%%%%%%%%%%%%%%%%%%%%%%%%%%%  Acknowledgements %%%%%%%%%%%%%%%%%%%%%%%%%%%%%%

\intro{Acknowledgements}

First and foremost I would like to thank to Dr. Andrea Bracciali for introducing me to blockchain. What's more I am grateful for his patience, guidance and support, without which I would not have been able to grow into what I am today. Thank you.

I am grateful to my parents for giving me the opportunity to study in University of Stirling. 

I am grateful to my classmates for helping me out, when I could not understand something, and just for being around, it was fun. 

I would like to thank the library for having the nice, quiet, fourth floor, and for always buying the books I requested. 

%%%%%%%%%%%%%%%%%%%%%%%%%%%%%%%%%%  Contents %%%%%%%%%%%%%%%%%%%%%%%%%%%%%%%%%%%

\tableofcontents

\listoffigures                  % Uncomment as required

% \listoftables                 % Uncomment as required

\clearpage

\pagenumbering{arabic}

\setcounter{page}{1}


%%%%%%%%%%%%%%%%%%%%%%%%%%%%%%%%  Introduction %%%%%%%%%%%%%%%%%%%%%%%%%%%%%%%%%

\chapter{Introduction}

\label{introduction}

Chapters are entered using the \textit{{\bs}chapter} command; they start on a
new page. Ordinary text is presented in 11 point Times, single-spaced,
single-sided pages. In general, use the default spacing that headings and
paragraphs give you. If you need to use quotes, preferably use single curly
quotes `\ldots'.

\textit{Italics text} uses the \textit{{\bs}textit} command, while \textbf{bold
text} uses the \textit{{\bs}textbf} command. Chapters, sections, etc.\ are
identified by the \textit{{\bs}label} command, and are referenced by the
\textit{{\bs}ref} command (e.g.\ chapter~\ref{conclusion}).

\section{Background and Context}

\label{introduction-background}

Give the background to your project and context of what you have done. Sections
are entered using the \textit{{\bs}section} command.

\section{Scope and Objectives}

\label{introduction-objectives}

Define the scope and objectives of your project.

\section{Achievements}

\label{introduction-achievements}

Summarise what you have achieved.

\section{Overview of Dissertation}

Briefly overview the contents of what follows in the dissertation.


%%%%%%%%%%%%%%%%%%%%%%%%%%%%%%  State-of-The-Art %%%%%%%%%%%%%%%%%%%%%%%%%%%%%%%

\chapter{State-of-The-Art}

\label{state}
Summarise current knowledge and what others have done in the various topics of
your dissertation. Write for someone familiar with computing, but not
necessarily expert in the particular topics of your project.

It is important to write a {\em critical} literature review that identifies gaps
in current solutions and that clearly shows how the project was driven to
address these gaps. This chapter  should therefore feed into well-defined
requirements for the project. Avoid a banal description of related work that
does not carefully analyse its strengths and weaknesses.

Give references to other work by using citations like
\cite{Ji-Turner-1999c}. Use the \textit{{\bs}cite} command to cite references.
Books \cite{Greene-Williams-1997}, standards \cite{ISO-8807}, reports
\cite{Jacobson-Andersen-1997}, journal articles \cite{Turner-2007a}, conference
papers \cite{Ji-Turner-1999c}, and web pages \cite{Turner-2000a} are
conventionally presented in slightly different ways. If a web page does not have
a date, you should give the date on which you consulted it.

Citations are created with a {\it thebibliography} environment and
\textit{{\bs}bibitem} commands in a `.bbl' file. Unless you are willing to
invest time in creating Bib\TeX\ bibliographies, you can do this by hand.

%%%%%%%%%%%%%%%%%%%%%%%%%%%%%% Technical Chapters %%%%%%%%%%%%%%%%%%%%%%%%%%%%%%

\chapter{Technical Chapters}            % Insert your chapter titles

\label{technical}

The body of the dissertation consists of a number of chapters named
appropriately ({\em not} `Technical Chapter'). Follow a logical progression in
how you present your work. This might be a time sequence of development
activities, the phases of the software development cycle, the modules of your
system, etc.

Appropriate chapters might be called Requirements, Design, Implementation and
Evaluation. The emphasis should be on requirements, design and evaluation, with
implementation details being of lesser importance. The requirements should be
clearly stated, following from the client needs and weaknesses identified in the
state-of-the-art review. The design should include discussion of the choices
that were available and why particular decisions were made. The evaluation
should relate back to the requirements, and demonstrate the extent to which
these were met. Low-level material should appear in appendixes.

\section{First Section}

\label{technical-section1}

Subdivide your text into sections with the \textit{{\bs}section} command.

\subsection{First Subsection}

\label{technical-subsection1}

If necessary, also use subsections. Subsections are entered using the
\textit{{\bs}subsection} command.

\subsubsection{First Subsubsection}

\label{technical-subsubsection1}

If you really need subsubsections, enter these using the
\textit{{\bs}subsubsection} command.

\subsubsection{Second Subsubsection}

\label{technical-subsubsection2}

And yet more subsubsections if need be.

\subsection{Second Subsection}

\label{technical-subsection2}

And, as required, more subsections.

\section{Second Section}

\label{technical-section2}

Figures are created with the \textit{figure} environment, while tables are
created with the \textit{table} environment. They are identified by the
\textit{{\bs}label} command, and are referenced by the \textit{{\bs}ref}
command. Graphics are inserted with the \textit{{\bs}graphic} command. Captions
are entered using the \textit{{\bs}caption} command. As an example of a figure,
consider figure~\ref{figure}.

The native format for \LaTeX\ graphics is EPS (Encapsulated PostScript).
Graphical editors are usually capable of producing EPS. When outputting to PDF
(Portable Document Format), the native graphics format is also PDF. Conversion
of EPS to PDF is supported by a number of TeX\ toolsets.

\begin{figure}

  \graphic{figure}

  \caption{Highly Technical Diagram}

  \label{figure}

\end{figure}


%%%%%%%%%%%%%%%%%%%%%%%%%%%%%%%%%%  Conclusion %%%%%%%%%%%%%%%%%%%%%%%%%%%%%%%%%

\chapter{Conclusion}

\label{conclusion}

\section{Evaluation}

\label{conclusion-evaluation}

If you do not have a separate chapter on testing, explain here in detail how you
went about systematically testing your system. If appropriate, also include
end users in your testing. Summarise your main results, and explain how you have
advanced the state-of-the-art. Stand back and evaluate what you have achieved
and how well you have met the objectives. Evaluate your achievements against the
objectives stated in section~\ref{introduction-objectives}. Demonstrate that you
have tackled the project in a professional manner.

\section{Future Work}

\label{conclusion-future}

Explain any limitations in your results and how things might be improved.
Discuss how your work might be developed further. Reflect on your results in
isolation and in relation to what others have achieved in the same field. This
self-analysis is particularly important. You should give a critical evaluation
of what went well, and what might be improved.

% Citations

\bibliographystyle{abbrv}

\bibliography{dissertation}

% Appendixes

\appendix


%%%%%%%%%%%%%%%%%%%%%%%%%%%%%%%%%%  User Guide %%%%%%%%%%%%%%%%%%%%%%%%%%%%%%%%%

\chapter{User Guide}			% Insert your appendix titles

\label{guide}

The appendixes should contain reference material or detailed material that would
detract from the flow in the body of the dissertation. Appendixes might include
a user guide, a list of abbreviations, detailed program descriptions, etc.
Appendixes are introduced with the \textit{{\bs}appendix} command. Appendix
headings otherwise use \textit{{\bs}chapter}, \textit{{\bs}section}, etc.\ as
usual.

\end{document}
